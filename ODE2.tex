% Setting up the document class for a standard article
\documentclass[a4paper,12pt]{article}

% Including essential packages for mathematical typesetting and structure
\usepackage{amsmath} % For advanced math environments
\usepackage{amssymb} % For additional math symbols
\usepackage{geometry} % For page layout control
\geometry{margin=1in} % Setting 1-inch margins
\usepackage{parskip} % For paragraph spacing instead of indentation

% Including font package last for proper configuration
\usepackage{noto} % Using Noto font for compatibility

% Beginning the document
\begin{document}

% Creating a title for the document
\title{Solution to the Differential Equation \(mg - \beta \dot{x}^2 - k(x - x_0) = m \ddot{x}\)}
\author{}
\date{}
\maketitle

% Starting the solution section
\section*{Solution with Initial Conditions \(x(0) = x_0\), \(\dot{x}(0) = v_0\)}

Given the differential equation \(mg - \beta \dot{x}^2 - k(x - x_0) = m \ddot{x}\) with initial conditions \(x(0) = x_0\) and \(\dot{x}(0) = v_0\), where \(m\), \(g\), \(\beta\), \(k\), and \(x_0\) are constants, \(\dot{x} = \frac{dx}{dt}\), and \(\ddot{x} = \frac{d^2x}{dt^2}\), we solve as follows. The equation is nonlinear due to the \(\beta \dot{x}^2\) term, complicating analytical solutions.

% Step 1: Rewriting and simplifying
\subsection*{Step 1: Rewrite and Simplify}
The equation is:
\begin{equation}
m \ddot{x} + \beta \dot{x}^2 + k(x - x_0) = mg
\end{equation}
Substitute \(y = x - x_0\), so \(x = y + x_0\), \(\dot{x} = \dot{y}\), \(\ddot{x} = \ddot{y}\):
\begin{equation}
m \ddot{y} + \beta \dot{y}^2 + k y = mg
\end{equation}
The initial conditions become:
\[
y(0) = x(0) - x_0 = x_0 - x_0 = 0, \quad \dot{y}(0) = \dot{x}(0) = v_0
\]

% Step 2: Equilibrium analysis
\subsection*{Step 2: Equilibrium Analysis}
Set \(\ddot{y} = \dot{y} = 0\):
\[
k y = mg \implies y = \frac{mg}{k}
\]
Thus, the equilibrium position is \(x = x_0 + \frac{mg}{k}\), where the spring force balances gravity.

% Step 3: Attempt analytical solution
\subsection*{Step 3: Attempt Analytical Solution}
The nonlinear term \(\beta \dot{y}^2\) makes analytical solutions challenging. We explore homogeneous and particular solutions.

\subsubsection*{Homogeneous Equation}
For \(mg = 0\) and \(\beta = 0\):
\begin{equation}
m \ddot{y} + k y = 0
\end{equation}
\[
\ddot{y} + \omega^2 y = 0, \quad \omega = \sqrt{\frac{k}{m}}
\]
The solution is:
\begin{equation}
y_h(t) = c_1 \cos(\omega t) + c_2 \sin(\omega t)
\end{equation}
However, the \(\beta \dot{y}^2\) term prevents direct use.

\subsubsection*{Particular Solution}
Try \(y_p = A\) (constant):
\[
\ddot{y}_p = 0, \quad \dot{y}_p = 0
\]
\[
k A = mg \implies A = \frac{mg}{k}
\]
\[
y_p(t) = \frac{mg}{k}, \quad x_p(t) = x_0 + \frac{mg}{k}
\]
Verify:
\[
m \cdot 0 + \beta \cdot 0 + k \cdot \frac{mg}{k} = mg
\]
This satisfies the equation.

% Step 4: Phase-plane analysis
\subsection*{Step 4: Phase-Plane Analysis}
Set \(\dot{y} = v\), so \(\ddot{y} = \frac{dv}{dt} = v \frac{dv}{dy}\):
\begin{equation}
m v \frac{dv}{dy} + \beta v^2 + k y = mg
\end{equation}
\[
v \frac{dv}{dy} = \frac{mg - \beta v^2 - k y}{m}
\]
Or in time:
\begin{equation}
\frac{dv}{dt} = \frac{mg - \beta v^2 - k y}{m}
\end{equation}
With \(\frac{dy}{dt} = v\), we have:
\begin{align}
\frac{dy}{dt} &= v \\
\frac{dv}{dt} &= \frac{mg - \beta v^2 - k y}{m}
\end{align}
Initial conditions: \(y(0) = 0\), \(v(0) = v_0\). This system is nonlinear and typically requires numerical methods.

% Step 5: Apply initial conditions
\subsection*{Step 5: Apply Initial Conditions}
The particular solution \(y_p = \frac{mg}{k}\) gives \(y(0) = \frac{mg}{k} \neq 0\), which doesn’t satisfy the initial condition. The nonlinear term requires a dynamic solution.

% Step 6: Terminal velocity
\subsection*{Step 6: Terminal Velocity}
Assume \(\ddot{y} = 0\), \(\dot{y} = v_t\):
\[
\beta v_t^2 + k y = mg
\]
At equilibrium \(y = \frac{mg}{k}\):
\[
\beta v_t^2 = 0 \implies v_t = 0
\]
This suggests no terminal velocity at equilibrium.

% Step 7: Linear approximation
\subsection*{Step 7: Linear Approximation}
For small \(\beta\), approximate \(\beta \dot{y}^2 \approx 0\):
\begin{equation}
m \ddot{y} + k y \approx mg
\end{equation}
\[
\ddot{y} + \omega^2 y = g, \quad \omega = \sqrt{\frac{k}{m}}
\]
Particular solution: \(y_p = \frac{mg}{k}\). Homogeneous solution:
\[
y_h = c_1 \cos(\omega t) + c_2 \sin(\omega t)
\]
General solution:
\begin{equation}
y(t) = c_1 \cos(\omega t) + c_2 \sin(\omega t) + \frac{mg}{k}
\end{equation}
Apply initial conditions:
\[
y(0) = 0 = c_1 + \frac{mg}{k} \implies c_1 = -\frac{mg}{k}
\]
\[
\dot{y}(t) = -c_1 \omega \sin(\omega t) + c_2 \omega \cos(\omega t), \quad \dot{y}(0) = v_0 = c_2 \omega \implies c_2 = \frac{v_0}{\omega}
\]
\[
y(t) = -\frac{mg}{k} \cos(\omega t) + \frac{v_0}{\omega} \sin(\omega t) + \frac{mg}{k}
\]
\[
x(t) = x_0 + \frac{mg}{k} \left(1 - \cos(\omega t)\right) + \frac{v_0}{\omega} \sin(\omega t), \quad \omega = \sqrt{\frac{k}{m}}
\]

% Step 8: Numerical solution
\subsection*{Step 8: Numerical Solution}
The nonlinear system:
\begin{align}
\frac{dy}{dt} &= v \\
\frac{dv}{dt} &= \frac{mg - \beta v^2 - k y}{m}
\end{align}
with \(y(0) = 0\), \(v(0) = v_0\) requires numerical methods (e.g., Runge-Kutta) for an exact solution.

% Final answer
\subsection*{Final Answer}
The differential equation \(mg - \beta \dot{x}^2 - k(x - x_0) = m \ddot{x}\) with \(x(0) = x_0\), \(\dot{x}(0) = v_0\) is nonlinear. In terms of \(y = x - x_0\):
\begin{equation}
m \ddot{y} + \beta \dot{y}^2 + k y = mg, \quad y(0) = 0, \quad \dot{y}(0) = v_0
\end{equation}
\begin{itemize}
    \item \textbf{Approximate solution} (small \(\beta\)):
    \begin{equation}
    x(t) \approx x_0 + \frac{mg}{k} \left(1 - \cos\left(\sqrt{\frac{k}{m}} t\right)\right) + \frac{v_0}{\sqrt{\frac{k}{m}}} \sin\left(\sqrt{\frac{k}{m}} t\right)
    \end{equation}
    \item \textbf{Exact solution}: Solve the system:
    \begin{align}
    \frac{dy}{dt} &= v \\
    \frac{dv}{dt} &= \frac{mg - \beta v^2 - k y}{m}
    \end{align}
    numerically with \(y(0) = 0\), \(v(0) = v_0\). The behavior depends on \(m\), \(\beta\), \(k\), \(g\), and \(v_0\).
\end{itemize}

% Ending the document
\end{document}