% Setting up the document class for a standard article
\documentclass[a4paper,12pt]{article}

% Including essential packages for mathematical typesetting and structure
\usepackage{amsmath} % For advanced math environments
\usepackage{amssymb} % For additional math symbols
\usepackage{geometry} % For page layout control
\geometry{margin=1in} % Setting 1-inch margins
\usepackage{parskip} % For paragraph spacing instead of indentation

% Including font package last for proper configuration
\usepackage{noto} % Using Noto font for compatibility

% Beginning the document
\begin{document}

% Creating a title for the document
\title{Solution to the Differential Equation \(\ddot{x} + \beta \dot{x} = g\)}
\author{}
\date{}
\maketitle

% Starting the solution section
\section*{Solution}

To solve the differential equation \(\ddot{x} + \beta \dot{x} = g\), where \(g\) is a constant, \(\dot{x} = \frac{dx}{dt}\), and \(\ddot{x} = \frac{d^2x}{dt^2}\), we recognize it as a second-order linear ordinary differential equation with constant coefficients and a constant forcing term. Below is a step-by-step solution.

% Step 1: Rewriting the equation
\subsection*{Step 1: Rewrite the Equation}
The equation is:
\begin{equation}
\ddot{x} + \beta \dot{x} = g
\end{equation}
This can be written in standard form:
\begin{equation}
\ddot{x} + \beta \dot{x} - g = 0
\end{equation}
The left-hand side involves derivatives, and the right-hand side is a constant, indicating a non-homogeneous linear ODE.

% Step 2: Solving the homogeneous equation
\subsection*{Step 2: Solve the Homogeneous Equation}
First, solve the associated homogeneous equation:
\begin{equation}
\ddot{x} + \beta \dot{x} = 0
\end{equation}
Assume a solution of the form \(x(t) = e^{rt}\). Substituting gives the characteristic equation:
\begin{equation}
r^2 + \beta r = 0
\end{equation}
\[
r(r + \beta) = 0
\]
The roots are:
\[
r = 0 \quad \text{or} \quad r = -\beta
\]
Since the roots are real and distinct, the homogeneous solution is:
\begin{equation}
x_h(t) = c_1 + c_2 e^{-\beta t}
\end{equation}
where \(c_1\) and \(c_2\) are arbitrary constants.

% Step 3: Finding a particular solution
\subsection*{Step 3: Find a Particular Solution}
For the non-homogeneous equation \(\ddot{x} + \beta \dot{x} = g\), since \(g\) is a constant and \(r = 0\) is a root of the characteristic equation, try a particular solution of the form:
\begin{equation}
x_p(t) = A t
\end{equation}
Then:
\[
\dot{x}_p = A, \quad \ddot{x}_p = 0
\]
Substitute into the differential equation:
\[
0 + \beta \cdot A = g
\]
\[
A = \frac{g}{\beta} \quad (\beta \neq 0)
\]
Thus, the particular solution is:
\begin{equation}
x_p(t) = \frac{g}{\beta} t
\end{equation}

% Step 4: General solution
\subsection*{Step 4: General Solution}
The general solution is the sum of the homogeneous and particular solutions:
\begin{equation}
x(t) = x_h(t) + x_p(t) = c_1 + c_2 e^{-\beta t} + \frac{g}{\beta} t
\end{equation}

% Step 5: Case when beta = 0
\subsection*{Step 5: Consider the Case \(\beta = 0\)}
If \(\beta = 0\), the equation becomes:
\begin{equation}
\ddot{x} = g
\end{equation}
Integrate twice:
\[
\dot{x} = \int g \, dt = g t + c_1
\]
\[
x = \int (g t + c_1) \, dt = \frac{1}{2} g t^2 + c_1 t + c_2
\]
So, for \(\beta = 0\):
\begin{equation}
x(t) = \frac{1}{2} g t^2 + c_1 t + c_2
\end{equation}

% Step 6: Interpreting the solution
\subsection*{Step 6: Interpret the Solution}
\begin{itemize}
    \item For \(\beta \neq 0\), the solution \(x(t) = c_1 + c_2 e^{-\beta t} + \frac{g}{\beta} t\) includes:
    \begin{itemize}
        \item A constant term \(c_1\).
        \item An exponential term \(c_2 e^{-\beta t}\), which decays if \(\beta > 0\) or grows if \(\beta < 0\).
        \item A linear term \(\frac{g}{\beta} t\), representing the effect of the constant forcing \(g\).
    \end{itemize}
    \item For \(\beta = 0\), the solution is quadratic, indicating unbounded growth.
    \item Constants \(c_1\) and \(c_2\) are determined by initial conditions (e.g., \(x(0)\) and \(\dot{x}(0)\)).
\end{itemize}

% Final answer
\subsection*{Final Answer}
The solution to the differential equation \(\ddot{x} + \beta \dot{x} = g\) is:
\begin{itemize}
    \item If \(\beta \neq 0\):
    \begin{equation}
    x(t) = c_1 + c_2 e^{-\beta t} + \frac{g}{\beta} t
    \end{equation}
    \item If \(\beta = 0\):
    \begin{equation}
    x(t) = \frac{1}{2} g t^2 + c_1 t + c_2
    \end{equation}
\end{itemize}
where \(c_1\) and \(c_2\) are constants determined by initial conditions.

% Ending the document
\end{document}