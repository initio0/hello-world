% Setting up the document class for a standard article
\documentclass[a4paper,12pt]{article}

% Including essential packages for mathematical typesetting and structure
\usepackage{amsmath} % For advanced math environments
\usepackage{amssymb} % For additional math symbols
\usepackage{geometry} % For page layout control
\geometry{margin=1in} % Setting 1-inch margins
\usepackage{parskip} % For paragraph spacing instead of indentation

% Including font package last for proper configuration
\usepackage{noto} % Using Noto font for compatibility

% Beginning the document
\begin{document}

% Creating a title for the document
\title{Overview of Runge-Kutta Methods}
\author{}
\date{}
\maketitle

% Starting the explanation section
\section*{Runge-Kutta Methods}

Runge-Kutta (RK) methods are a family of numerical techniques for solving ordinary differential equations (ODEs) of the form:
\begin{equation}
\frac{d\mathbf{y}}{dt} = \mathbf{f}(t, \mathbf{y}), \quad \mathbf{y}(t_0) = \mathbf{y}_0
\end{equation}
where \(\mathbf{y}\) is a vector (for systems of ODEs) or scalar (for a single ODE), \(\mathbf{f}(t, \mathbf{y})\) defines the dynamics, and \(\mathbf{y}_0\) is the initial condition. These methods approximate the solution at discrete times \(t_{n+1} = t_n + h\), where \(h\) is the step size, by combining multiple slope evaluations.

% General framework
\subsection*{General Framework}
An \(s\)-stage Runge-Kutta method approximates \(\mathbf{y}_{n+1}\) from \(\mathbf{y}_n\) as:
\begin{equation}
\mathbf{y}_{n+1} = \mathbf{y}_n + h \sum_{i=1}^s b_i \mathbf{k}_i
\end{equation}
where the stage slopes are:
\begin{equation}
\mathbf{k}_i = \mathbf{f}\left(t_n + c_i h, \mathbf{y}_n + h \sum_{j=1}^s a_{ij} \mathbf{k}_j\right)
\end{equation}
The coefficients \(a_{ij}\), \(b_i\), and \(c_i\) are defined in a Butcher tableau:
\[
\begin{array}{c|cccc}
c_1 & a_{11} & a_{12} & \cdots & a_{1s} \\
c_2 & a_{21} & a_{22} & \cdots & a_{2s} \\
\vdots & \vdots & \vdots & \ddots & \vdots \\
c_s & a_{s1} & a_{s2} & \cdots & a_{ss} \\
\hline
& b_1 & b_2 & \cdots & b_s
\end{array}
\]
Explicit methods have \(a_{ij} = 0\) for \(j \geq i\), while implicit methods require iterative solutions.

% Common RK methods
\subsection*{Common Runge-Kutta Methods}

% Euler's method
\subsubsection*{First-Order (Euler’s Method)}
Single stage (\(s=1\)):
\begin{equation}
\mathbf{y}_{n+1} = \mathbf{y}_n + h \mathbf{f}(t_n, \mathbf{y}_n)
\end{equation}
Error per step: \(O(h^2)\), global error: \(O(h)\). Simple but less accurate.

% RK2
\subsubsection*{Second-Order RK (RK2)}
Example (midpoint method):
\begin{align}
\mathbf{k}_1 &= \mathbf{f}(t_n, \mathbf{y}_n) \\
\mathbf{k}_2 &= \mathbf{f}(t_n + h, \mathbf{y}_n + h \mathbf{k}_1) \\
\mathbf{y}_{n+1} &= \mathbf{y}_n + \frac{h}{2} (\mathbf{k}_1 + \mathbf{k}_2)
\end{align}
Error per step: \(O(h^3)\), global error: \(O(h^2)\).

% RK4
\subsubsection*{Fourth-Order RK (RK4)}
Four stages:
\begin{align}
\mathbf{k}_1 &= \mathbf{f}(t_n, \mathbf{y}_n) \\
\mathbf{k}_2 &= \mathbf{f}\left(t_n + \frac{h}{2}, \mathbf{y}_n + \frac{h}{2} \mathbf{k}_1\right) \\
\mathbf{k}_3 &= \mathbf{f}\left(t_n + \frac{h}{2}, \mathbf{y}_n + \frac{h}{2} \mathbf{k}_2\right) \\
\mathbf{k}_4 &= \mathbf{f}(t_n + h, \mathbf{y}_n + h \mathbf{k}_3) \\
\mathbf{y}_{n+1} &= \mathbf{y}_n + \frac{h}{6} ( \mathbf{k}_1 + 2 \mathbf{k}_2 + 2 \mathbf{k}_3 + \mathbf{k}_4 )
\end{align}
Error per step: \(O(h^5)\), global error: \(O(h^4)\). Widely used for its accuracy and simplicity.

% Higher-order and adaptive methods
\subsubsection*{Higher-Order and Adaptive Methods}
Methods like RK5 or Runge-Kutta-Fehlberg (RKF45) use more stages or adaptive step sizes. Implicit methods (e.g., Gauss-Legendre RK) handle stiff systems.

% Application to specific equation
\subsection*{Application to the Differential Equation}
For the equation \(mg - \beta \dot{x}^2 - k(x - x_0) = m \ddot{x}\), we use the substitution \(y = x - x_0\):
\begin{equation}
m \ddot{y} + \beta \dot{y}^2 + k y = mg
\end{equation}
Define \(y_1 = y\), \(y_2 = \dot{y}\):
\begin{align}
\frac{dy_1}{dt} &= y_2 \\
\frac{dy_2}{dt} &= \frac{mg - \beta y_2^2 - k y_1}{m}
\end{align}
\[
\mathbf{y} = \begin{bmatrix} y_1 \\ y_2 \end{bmatrix}, \quad \mathbf{f}(t, \mathbf{y}) = \begin{bmatrix} y_2 \\ \frac{mg - \beta y_2^2 - k y_1}{m} \end{bmatrix}
\]
Initial conditions: \(y_1(0) = 0\), \(y_2(0) = v_0\). RK4 iteratively applies the four-stage update to compute \(\mathbf{y}(t)\), yielding \(x(t) = y_1(t) + x_0\), \(\dot{x}(t) = y_2(t)\).

% Advantages and limitations
\subsection*{Advantages and Limitations}
\begin{itemize}
    \item \textbf{Advantages}:
    \begin{itemize}
        \item High accuracy (e.g., RK4 has \(O(h^4)\) global error).
        \item Easy to implement for non-stiff systems.
        \item Flexible for systems of ODEs.
    \end{itemize}
    \item \textbf{Limitations}:
    \begin{itemize}
        \item Fixed-step methods are inefficient for stiff systems.
        \item Implicit methods are stable but computationally intensive.
        \item Higher-order methods increase computational cost.
    \end{itemize}
\end{itemize}

% Practical considerations
\subsection*{Practical Considerations}
\begin{itemize}
    \item \textbf{Step Size (\(h\))}: Smaller \(h\) improves accuracy but increases computation. Adaptive methods adjust \(h\) dynamically.
    \item \textbf{Stability}: RK4 is stable for small \(h\); stiff systems (e.g., large \(k/m\)) may require implicit or adaptive methods.
    \item \textbf{Implementation}: RK4 is straightforward to code using libraries like NumPy.
\end{itemize}

% Final answer
\subsection*{Summary}
Runge-Kutta methods solve ODEs \(\frac{d\mathbf{y}}{dt} = \mathbf{f}(t, \mathbf{y})\) by combining multiple slope evaluations. The RK4 method, with four stages, offers \(O(h^4)\) global error and is applied to the system:
\begin{align}
\dot{y}_1 &= y_2 \\
\dot{y}_2 &= \frac{mg - \beta y_2^2 - k y_1}{m}
\end{align}
to numerically solve the given equation, providing accurate position and velocity solutions.

% Ending the document
\end{document}