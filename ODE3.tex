% Setting up the document class for a standard article
\documentclass[a4paper,12pt]{article}

% Including essential packages for mathematical typesetting and structure
\usepackage{amsmath} % For advanced math environments
\usepackage{amssymb} % For additional math symbols
\usepackage{geometry} % For page layout control
\geometry{margin=1in} % Setting 1-inch margins
\usepackage{parskip} % For paragraph spacing instead of indentation
\usepackage{listings} % For code listings
\lstset{
    basicstyle=\ttfamily\small,
    breaklines=true,
    frame=single,
    language=Python,
    numbers=left,
    numberstyle=\tiny,
    showstringspaces=false
}

% Including font package last for proper configuration
\usepackage{noto} % Using Noto font for compatibility

% Beginning the document
\begin{document}

% Creating a title for the document
\title{Numerical Solution to the Differential Equation \(mg - \beta \dot{x}^2 - k(x - x_0) = m \ddot{x}\)}
\author{}
\date{}
\maketitle

% Starting the solution section
\section*{Numerical Solution with Initial Conditions \(x(0) = x_0\), \(\dot{x}(0) = v_0\)}

To solve the nonlinear differential equation \(mg - \beta \dot{x}^2 - k(x - x_0) = m \ddot{x}\) with initial conditions \(x(0) = x_0\) and \(\dot{x}(0) = v_0\), where \(m\), \(g\), \(\beta\), \(k\), and \(x_0\) are constants, \(\dot{x} = \frac{dx}{dt}\), and \(\ddot{x} = \frac{d^2x}{dt^2}\), we use a numerical approach due to the nonlinear \(\beta \dot{x}^2\) term. We transform the second-order equation into a system of first-order ODEs and apply the fourth-order Runge-Kutta (RK4) method.

% Step 1: Transform to first-order system
\subsection*{Step 1: Transform to a First-Order System}
Rewrite the equation:
\begin{equation}
m \ddot{x} + \beta \dot{x}^2 + k(x - x_0) = mg
\end{equation}
Substitute \(y = x - x_0\), so \(x = y + x_0\), \(\dot{x} = \dot{y}\), \(\ddot{x} = \ddot{y}\):
\begin{equation}
m \ddot{y} + \beta \dot{y}^2 + k y = mg
\end{equation}
Initial conditions:
\[
x(0) = x_0 \implies y(0) = 0, \quad \dot{x}(0) = v_0 \implies \dot{y}(0) = v_0
\]
Define:
\[
y_1 = y, \quad y_2 = \dot{y} = \frac{dy}{dt}
\]
Then:
\[
\dot{y}_1 = y_2
\]
\[
\dot{y}_2 = \ddot{y} = \frac{mg - \beta \dot{y}^2 - k y}{m} = \frac{mg - \beta y_2^2 - k y_1}{m}
\]
The system is:
\begin{align}
\frac{dy_1}{dt} &= y_2 \\
\frac{dy_2}{dt} &= \frac{mg - \beta y_2^2 - k y_1}{m}
\end{align}
Initial conditions: \(y_1(0) = 0\), \(y_2(0) = v_0\). In vector form:
\[
\mathbf{y} = \begin{bmatrix} y_1 \\ y_2 \end{bmatrix}, \quad \frac{d\mathbf{y}}{dt} = \mathbf{f}(t, \mathbf{y}) = \begin{bmatrix} y_2 \\ \frac{mg - \beta y_2^2 - k y_1}{m} \end{bmatrix}
\]
This is an autonomous system since \(\mathbf{f}\) does not explicitly depend on \(t\).

% Step 2: Choose numerical method
\subsection*{Step 2: Choose a Numerical Method}
The RK4 method is used for its fourth-order accuracy. For \(\frac{d\mathbf{y}}{dt} = \mathbf{f}(t, \mathbf{y})\), the update from \(t_n\) to \(t_{n+1} = t_n + h\) is:
\begin{equation}
\mathbf{y}_{n+1} = \mathbf{y}_n + \frac{h}{6} ( \mathbf{k}_1 + 2 \mathbf{k}_2 + 2 \mathbf{k}_3 + \mathbf{k}_4 )
\end{equation}
where:
\begin{align}
\mathbf{k}_1 &= \mathbf{f}(t_n, \mathbf{y}_n) \\
\mathbf{k}_2 &= \mathbf{f}\left(t_n + \frac{h}{2}, \mathbf{y}_n + \frac{h}{2} \mathbf{k}_1\right) \\
\mathbf{k}_3 &= \mathbf{f}\left(t_n + \frac{h}{2}, \mathbf{y}_n + \frac{h}{2} \mathbf{k}_2\right) \\
\mathbf{k}_4 &= \mathbf{f}(t_n + h, \mathbf{y}_n + h \mathbf{k}_3)
\end{align}

% Step 3: Apply RK4
\subsection*{Step 3: Apply RK4 to the System}
For:
\[
\mathbf{f}(t, \mathbf{y}) = \begin{bmatrix} y_2 \\ \frac{mg - \beta y_2^2 - k y_1}{m} \end{bmatrix}
\]
Compute:
\begin{itemize}
    \item \(\mathbf{k}_1 = \begin{bmatrix} y_{2,n} \\ \frac{mg - \beta y_{2,n}^2 - k y_{1,n}}{m} \end{bmatrix}\)
    \item \(\mathbf{k}_2 = \mathbf{f}\left(t_n + \frac{h}{2}, \mathbf{y}_n + \frac{h}{2} \mathbf{k}_1\right)\)
    \item \(\mathbf{k}_3 = \mathbf{f}\left(t_n + \frac{h}{2}, \mathbf{y}_n + \frac{h}{2} \mathbf{k}_2\right)\)
    \item \(\mathbf{k}_4 = \mathbf{f}(t_n + h, \mathbf{y}_n + h \mathbf{k}_3)\)
\end{itemize}
Update:
\[
\mathbf{y}_{n+1} = \mathbf{y}_n + \frac{h}{6} ( \mathbf{k}_1 + 2 \mathbf{k}_2 + 2 \mathbf{k}_3 + \mathbf{k}_4 )
\]

% Step 4: Python implementation
\subsection*{Step 4: Python Implementation}
Below is a Python script using NumPy to implement RK4, with example parameters.

\begin{lstlisting}
import numpy as np
import matplotlib.pyplot as plt

# Parameters
m = 1.0    # mass (kg)
g = 9.81   # gravity (m/s^2)
beta = 0.5 # damping coefficient (kg/m)
k = 10.0   # spring constant (N/m)
x0 = 0.0   # reference position (m)
v0 = 1.0   # initial velocity (m/s)

# Time settings
t_max = 10.0  # total time (s)
h = 0.01      # step size (s)
n_steps = int(t_max / h)
t = np.linspace(0, t_max, n_steps + 1)

# Initialize arrays
y = np.zeros((n_steps + 1, 2))  # [y1, y2]
y[0] = [0.0, v0]               # initial conditions: y1(0) = 0, y2(0) = v0

# Define the ODE system
def f(t, y):
    y1, y2 = y
    return np.array([y2, (m*g - beta*y2**2 - k*y1) / m])

# RK4 solver
for n in range(n_steps):
    y_n = y[n]
    k1 = f(t[n], y_n)
    k2 = f(t[n] + h/2, y_n + (h/2) * k1)
    k3 = f(t[n] + h/2, y_n + (h/2) * k2)
    k4 = f(t[n] + h, y_n + h * k3)
    y[n + 1] = y_n + (h/6) * (k1 + 2*k2 + 2*k3 + k4)

# Convert back to x
x = y[:, 0] + x0
v = y[:, 1]

# Plot results
plt.figure(figsize=(10, 6))
plt.subplot(2, 1, 1)
plt.plot(t, x, label='Position $x(t)$')
plt.xlabel('Time (s)')
plt.ylabel('Position (m)')
plt.grid(True)
plt.legend()

plt.subplot(2, 1, 2)
plt.plot(t, v, label='Velocity $\dot{x}(t)$', color='orange')
plt.xlabel('Time (s)')
plt.ylabel('Velocity (m/s)')
plt.grid(True)
plt.legend()

plt.tight_layout()
plt.show()
\end{lstlisting}

% Step 5: Interpret the solution
\subsection*{Step 5: Interpret the Numerical Solution}
\begin{itemize}
    \item \textbf{Parameters}: Example values: \(m = 1\), \(g = 9.81\), \(\beta = 0.5\), \(k = 10\), \(x_0 = 0\), \(v_0 = 1\). Adjust based on the physical system.
    \item \textbf{Step Size}: \(h = 0.01\) ensures accuracy; smaller \(h\) improves precision.
    \item \textbf{Output}: The script computes \(y_1(t) = y(t) = x(t) - x_0\) and \(y_2(t) = \dot{x}(t)\), converting to \(x(t) = y_1(t) + x_0\). Plots show position and velocity.
    \item \textbf{Behavior}: The quadratic damping \(\beta \dot{y}^2\) causes the system to approach equilibrium \(y = \frac{mg}{k}\) (or \(x = x_0 + \frac{mg}{k}\)) with damped oscillations, depending on \(v_0\) and \(\beta\).
\end{itemize}

% Step 6: Considerations
\subsection*{Step 6: Considerations}
\begin{itemize}
    \item \textbf{Stability}: RK4 is stable for small \(h\). For stiff systems (large \(k/m\)), consider adaptive methods (e.g., \texttt{scipy.integrate.solve\_ivp}).
    \item \textbf{Accuracy}: Verify by reducing \(h\) and checking convergence.
    \item \textbf{Physical Interpretation}: The system models a mass on a spring with quadratic damping and gravity, capturing damped oscillations or monotonic approaches to equilibrium.
\end{itemize}

% Final answer
\subsection*{Final Answer}
To numerically solve \(mg - \beta \dot{x}^2 - k(x - x_0) = m \ddot{x}\) with \(x(0) = x_0\), \(\dot{x}(0) = v_0\):
\begin{enumerate}
    \item Transform to:
    \begin{align}
    \frac{dy_1}{dt} &= y_2 \\
    \frac{dy_2}{dt} &= \frac{mg - \beta y_2^2 - k y_1}{m}
    \end{align}
    with \(y_1(0) = 0\), \(y_2(0) = v_0\).
    \item Apply RK4 with a small step size (e.g., \(h = 0.01\)).
    \item Implement in Python (as shown) to compute \(x(t) = y_1(t) + x_0\) and \(\dot{x}(t) = y_2(t)\).
    \item Analyze results to understand the system’s behavior, influenced by \(m\), \(\beta\), \(k\), \(g\), and \(v_0\).
\end{enumerate}

% Ending the document
\end{document}